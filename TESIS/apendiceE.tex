\chapter{Uso del programa en Mathematica para el cálculo de Hamiltonianos efectivos cuánticos}

Los cálculos de Hamiltonianos efectivos por el método iterativo de Floquet-Magnus se pueden modelar en un programa computacional usando, en nuestro caso, \textit{Mathematica} y obtener resultados fiables para cualquier potencial perturbativo externo al electrón en la red de \textit{Enlace Fuerte}. En este apéndice presentamos el código escrito y su uso para hallar el operador Hamiltoniano efectivo  $\hat{G}$ y el operador de micromovimiento $\hat{P}$ hasta orden 5 para el caso del Hamiltoniano estudiado por Martínes (2017) \cite{martinez2017}
Sea $H$ el Hamiltoniano del electrón en el modelo de \textit{Enlace Fuerte} en presencia de perturbaciones $V_1(x)$ y $V_2(x,t)$

\begin{equation}
    H=-2A\cos(ak)+V_1(x)+V_2(x,t)
\end{equation}

Se definine $H_0$ como el Hamiltoniano de \textit{Tight Binding} en presencia de la perturbación independiente del tiempo $V_1$

\begin{equation}
    H_o=-2A\cos(ak)+ V_1(x)
\end{equation}

Y sea $H_1(x,t)$ la perturbación dependiente del tiempo, en el caso particular de este trabajo de grado, se estudian solamente perturbaciones sinusoidales.

\begin{equation}
    H_1= V_2(x,t)=V_2(x)\cos(\omega t)
\end{equation}
 
Donde $V_2$ es una función dependiente solo de $x$, llamaremos a la perturbación total $V(x,t)$.

\begin{equation}
    V(x,t)=V_1(x)+V_2(x,t)
\end{equation}

Las funciones $V_1(x)$ y $V_2(x)$ pueden ser escritas como series polinomiales, escribiéndolas como series de Taylor centradas en cero. Se definirán $V_1(x)$ y $V_2(x)$ como operadores polinomio de orden $n$, para el caso de este programa será suficiente trabajar con polinomios hasta orden $n=4$.

\begin{equation}
    V_1(x) = a_o + a_1x + a_2x^2 + a_3x^3 + a_4x^4
\end{equation}

\begin{equation}
    V_2(x) = b_o + b_1x + b_2x^2 + b_3x^3 + b_4x^4
\end{equation}

Al trabajar las ecuaciones en forma cuántica, estas serán tratadas como operadores. En Mecánica cuántica el operador posición se define como sigue:

\begin{equation}
    x = i \hslash \frac{d}{dp} 
\end{equation}

donde se tomará la constante $\hslash = 1$.Entonces, los polinomios $V_1(x)$ y $V_2(x)$ tienen como operadores:  

\begin{equation}
    \hat{V_1} = a_o + a_1 i \frac{d}{dp} - a_2 \frac{d^2}{dp^2} - a_3 i \frac{d^3}{dp^3} + a_4\frac{d^4}{dp^4} 
\end{equation}

\begin{equation}
    \hat{V_2} = b_o + b_1 i \frac{d}{dp} - b_2 \frac{d^2}{dp^2} - b_3 i \frac{d^3}{dp^3} + b_4\frac{d^4}{dp^4} 
\end{equation}

Las fórmulas recursivas de Floquet-Magnus permiten hallar el Hamiltoniano efectivo para cada $V_1(x)$ y $V_2(x)$ que se quiera estudiar. 

\begin{equation}
    G^{(k)}=\frac{1}{T}\int^{T}_{0}\{ H(t')P^{(k-1)}(t')-\sum^{k-1}_{j=1} P^{(j)}G^{(k-j)}\}dt'
\end{equation}

Donde $T=2\pi/\omega$,es el periodo de la fuerza externa.

\begin{equation}
    P^{(k)}=-i\int^{t}_{0}\{ H(t')P^{(k-1)}(t')-\sum^{k-1}_{j=1} P^{(j)}G^{(k-j)}-G^{(k)}\}dt'
\end{equation}

Y por definición 

\begin{equation}
    \begin{cases}
        G^{(0)}=0\\
        P^{(0)}=1
    \end{cases}
\end{equation}

\begin{equation}
\begin{cases}
    G=\sum G^{(k)}\\
    P=\sum P^{(k)}
    \end{cases}
\end{equation}

\section{Uso del programa}

Se usará el programa desarrollado en Mathematica para reproducir resultados del Paper Martinez et al (2017)\cite{martinez2017}.  El Hamiltiniano tiene la forma siguiente,

\begin{equation}
    H(x,t)=-2A\cos(ap)+\epsilon x \cos(\omega t)
\end{equation}

\begin{itemize}
    \item Primero \textbf{se borra la memoria del kernel} para limpiar cualquier variable que se tenga cargada en la memoria del programa.

\begin{lstlisting}[language=Mathematica]
 ClearAll["Global`*"]
\end{lstlisting}

\item Se definen los coeficientes de $V_1(x)$ y $V_2(x)$ 

\begin{lstlisting}[language=Mathematica]
a0=0;
a1=0;
a2=0;
a3=0;
a4=0;
\end{lstlisting}

\begin{lstlisting}[language=Mathematica]
b0=0;
b1=eps;
b2=0;
b3=0;
b4=0;
\end{lstlisting}

 \item Se escriben $V_1(x)$ y $V_2(x)$ como funciones en Mathematica que trabajan como operadores $\hat{V_1}$ y $\hat{V_2}$ sobre cualquier función de $\rho$

\begin{lstlisting}[language=Mathematica]
V1[x_]:=a0+I*a1*D[x,{p,1}]-a2*D[x,{p,2}]-a3*I*D[x,{p,3}]+
a4*D[x,{p,4}]
\end{lstlisting}

\begin{lstlisting}[language=Mathematica]
V2[x_]:= b0+b1*I  D[x,{p,1}]-b2*D[x,{p,2}]-b3*I*D[x,{p,3}]+
b4*D[x,{p,4}]
\end{lstlisting}

\item Haciendo uso de estos operadores, se definen los operadores $H_o$,$H_1$ y $H$

\begin{lstlisting}[language=Mathematica]
Ho[x_]:=-2A Cos[a p]x+V1[x]  
H1[x_]:=V2[x]*Cos[ w t ]
H[x_]:=Ho[x]+H1[x]
\end{lstlisting}

\item Siguiendo las formulas recursivas de Floquet-Magnus se calculará con la ayuda de Mathematica los valores de $G^{(k)}$ hasta $k=5$


\begin{enumerate}

    \item \textbf{Cálculo de $G^{(0)}$ y $P^{(0)}$}
    
$G^{(0)}$ y $P^{(0)}$ se conocen a priori por lo que solo los definiremos en el entorno de trabajo de Mathematica,con la salvedad que estos deben definirse ccomo operadores.
    
\begin{equation}
    \begin{cases}
        G^{(0)}=0\\
        P^{(0)}=1
    \end{cases}
\end{equation}

\begin{lstlisting}[language=Mathematica]
G0[x_]:=0 ;
P0[x_]:=x ;
\end{lstlisting}

\item \textbf{Cálculo de $G^{(1)}$ y $P{(1)}$}

Usando la formula recursiva de Floquet-Magnus se tiene 

\begin{equation}
    G^{(1)}=\frac{1}{T}\int^{T}_{(0)} H(t')P^{(0)}dt'
\end{equation}

\begin{equation}
    P^{(1)}=-i\int^{t}_{(0)} H(t')P^{(0)}(t')-G^{(1)}dt'
\end{equation}

\begin{lstlisting}[language=Mathematica]
G1[x_]:=(w/(2Pi)) Integrate[ H@P0[x],{ t ,0, 2Pi/w } ]
P1[x_]:=-I Integrate[H@P0[x]-G1[x],{t,0,t}]
\end{lstlisting}

Se aplicaran los operadores $G^{(1)}$ y $P^{(1)}$ sobre una funcion arbitraria $f(p)$

\begin{lstlisting}
G1[f[p]]
P1[f[p]]
\end{lstlisting}

Obteniendo asi los resultados 
\begin{equation}
    G^{(1)}=-2A\cos(ap)=H_o
\end{equation}

\begin{equation}
    P^{(1)}=\frac{\epsilon\sin(\omega t)}{\omega}\mathbb{I}
\end{equation}

Este algoritmo se repetirá para cada $k$ hasta $k=5$ definiendo $G^{(k)}$ y $P^{(k)}$ apropiados para cada caso de acuerdo a la expansión de Floquet-Magnus.

\item \textbf{Cálculo de $G_2$ y $P_2$}

\begin{equation}
    G^{(2)}=\frac{1}{T}\int^{T}_{0} H(t')P^{1}(t')- P^{1}G^{1}dt'
\end{equation}

\begin{equation}
    P^{(2)}=-i\int^{t}_{0}\{ H(t')P^{1}(t')- P^{1}G^{1}-G^{2}\}dt'
\end{equation}

\begin{lstlisting}[language=Mathematica]
G2[x_]:=(w/(2Pi))* Integrate[H@P1[x]-P1@G1[x],{t,0,2Pi/w}]
\end{lstlisting}

\begin{lstlisting}[language=Mathematica]
P2[x_]:= -I* Integrate[H@P1[x]-P1@G1[x]-G2[x],{t,0,t}]
\end{lstlisting}



\begin{lstlisting}[language=Mathematica]
G2[f[p]]
0
P2[f[p]]
(1/(w^2))eps Sin[(t w)/2]^2 (4 I a A f[p] Sin[a p]+
eps (1+Cos[t w]) f''[p])
\end{lstlisting}

\begin{equation}
    G^{2}=0
\end{equation}

\begin{equation}
    P^{(2)}=\frac{\epsilon\sin^2(\frac{t\omega}{2})}{\omega^2}[4iaA\sin(ap)\mathbb{I}+2\epsilon\cos^2(\frac{t\omega}{2})\frac{\partial^2}{\partial p^2}]
\end{equation}

\item \textbf{Cálculo de $G_3$ y $P_3$}

\begin{equation}
    G^{3}=\frac{1}{T}\int^{T}_{0}\{ H(t')P^{2}(t')- P^{1}G^{2}-P^{2}G^{1}\}dt'
\end{equation}

\begin{equation}
    P^{(3)}=-i\int^{t}_{0}\{ H(t')P^{2}(t')- P^{1}G^{2}-P^{2}G^{1}-G^{3}\}dt'
\end{equation}

\begin{lstlisting}[language=Mathematica]
G3[x_]:=(w/(2Pi))*(Integrate[ H@P2[x]-P1@G2[x]-P2@G1[x],
{t,0,2Pi/w}])
\end{lstlisting}

\begin{lstlisting}
P3[x_]:=-I*(Integrate[ H@P2[x]-P1@G2[x]-P2@G1[x]-G3[x],
{t,0,t}]) 
\end{lstlisting}

\begin{equation}
    G^{(3)}=\frac{a^2A\epsilon^2\cos(ap)}{2\omega^2}\mathbb{I}
\end{equation}

\begin{equation}
         \begin{split}         P^{(3)}=&\frac{\epsilon^2}{12\omega^3}[3ia^2A(8\sin(\omega t)-3\sin(2\omega t))\cos(ap)\mathbb{I} +\\&+48iaA\sin^2(\frac{\omega t}{2})\sin(\omega t)\sin(ap)\frac{\partial}{\partial p}+\\&+ 2i\epsilon\sin^3(\omega t)\frac{\partial^3}{\partial p^3}]
    \end{split}
\end{equation}

\item \textbf{Cálculo de $G_4$ y $P_4$}

\begin{equation}
    G^{(4)}=\frac{1}{T}\int^{T}_{0}\{ H(t')P^{3}(t')- P^{1}G^{3}- P^{2}G^{2}- P^{3}G^{1}\}dt'
\end{equation}

\begin{equation}
    P^{(4)}=-i\int^{t}_{0}\{ H(t')P^{3}(t')- P^{1}G^{3}-P^{2}G^{2}-P^{3}G^{1}-G^{4}\}dt'
\end{equation}

\begin{lstlisting}[language=Mathematica]
G4[x_]:=(w/(2Pi))*(Integrate[ H@P3[x]-P1@G3[x]-P2@G2[x]-
P3@G1[x],{t,0,2Pi/w}])
\end{lstlisting}

\begin{lstlisting}[language=Mathematica]
P4[x_]:=-I*(Integrate[ H@P3[x]-P1@G3[x]-P2@G2[x]-P3@G1[x]-
G4[x],{t,0,t}]) 
\end{lstlisting}

\begin{lstlisting}
G4[f[p]]
P4[f[p]]
\end{lstlisting}


\begin{equation}
\begin{aligned}
        &P^{(4)}=\frac{\epsilon^2 \sin ^2\left(\frac{t \omega}{2}\right)}{36 \omega^4} (6 \epsilon^2  \sin ^2(t\omega) \cos^2 \left(\frac{t \omega}{2}\right)\frac{\partial^4}{\partial p^4}-\\&-72 i a^2 A \epsilon  \cos (a p) \cos
   ^2\left(\frac{t \omega}{2}\right) (3 \cos (t \omega)-4)\frac{\partial}{\partial p}+\\&+72 i a A \epsilon \sin (a p) \sin ^2(t \omega)\frac{\partial^2}{\partial p^2}-\\&-2 i a^2 A \sin (a p) (a \epsilon (14 \cos (t w)-11 \cos (2 t \omega)+33)+\\&+72 i A \sin (a p) (\cos
   (t \omega)-1))\mathbb{I})
    \end{aligned}
\end{equation}

\begin{equation}
    G^{(4)}=0
\end{equation}

\item \textbf{Cálculo de $G_5$ y $P_5$}

\begin{equation}
    G^{(5)}=\frac{1}{T}\int^{T}_{0}\{ H(t')P^{4}(t')- P^{1}G^{4}- P^{2}G^{3}- P^{3}G^{2}-P^{4}G^{1}\}dt'
\end{equation}

\begin{lstlisting}[language=Mathematica]
G5[x_]:=(w/(2Pi))*(Integrate[ H@P4[x]-P1@G4[x]-P2@G3[x]-
P3@G2[x]-P4@G1[x],{t,0,2Pi/w}])
G5[f[p]]
\end{lstlisting}


\begin{equation}
    G^{(5)}=-\frac{a^4 A \epsilon^4\cos(a p)}{32 \omega^4}\mathbb{I}
\end{equation}

\end{enumerate}

\end{itemize}

