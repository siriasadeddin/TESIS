\chapter*{Conclusiones}

 Se modeló la dinámica de un electrón en una red unidimensional, cuando sobre él actúan campos eléctricos externos inhomogéneos y rápidamente oscilantes, usando el modelo de \textit{Tight Binding}. La red utilizada es periódica, homonuclear, sin impurezas y restringiendo al electrón a mantenerse dentro de una sola banda. 
Se hizo uso del modelo semiclásico y el método de Kapitza para encontrar el potencial efectivo del electrón y de la teoría de Floquet y la expansión de Floquet-Magnus para obtener los  Hamiltonianos efectivos cuánticos logrando reproducir los resultados de Martínez et al. (2017) \cite{martinez2017} y Martínez et al. (2014) \cite{mart2014}, este último con una modificación en el orden de aproximación. 

Haciendo uso del enfoque semiclásico, se obtuvieron la forma del potencial efectivo, el Hamiltoniano efectivo y la masa efectiva del electrón; lo que permitió obtener un modelo para la dinámica en la red de \textit{Tight Binding} bajo la influencia de cualquier potencial externo, rápidamente oscilante o no. Se halló que en límite del continuo el potencial efectivo, el Hamiltoniano efectivo y la masa efectiva tienden al los resultados de Kapitza \cite{kapitza}, Landau \cite{landau} y Malay Bandyopadhyay et al. \cite{datta}.

Los resultados obtenidos fueron utilizados para modelar la dinámica del electrón en presencia de un potencial externo lineal y otro tipo tangente hiperbólica, ambos rápidamente oscilantes. Obteniendo que para el potencial lineal el electrón no presenta comportamiento oscilatorio, por ende, el electrón tendrá una dinámica con posición no acotada. En el caso de la tangente hiperbólica se halló que el potencial efectivo tiene dos puntos de equilibrio estable simétricos con el origen y uno inestable en el origen. Para los puntos de equilibrio estables se calculó la frecuencia de pequeñas oscilaciones $\omega_1$; adicionalmente se hizo un diagrama de fases para tener una referencia gráfica del comportamiento del electrón.

 En el enfoque cuántico, se halló una aproximación del Hamiltoniano efectivo independiente del tiempo, usando la expansión de Floquet-Magnus; esta aproximación se hizo hasta hasta ordenes de $1/\omega^2$, términos de corrección superiores no se tomaron en cuenta. Se encontró que la ecuación de Schrödinger para el Hamiltoniano efectivo es una ecuación diferencial de Hill. 
 
 Haciendo uso de esta aproximación se estudiaron nuevamente los casos del potencial lineal y la tangente hiperbólica. Se halló la ecuación de Schrödinger en cada caso y las respectivas autofunciones; para el caso lineal se encontró que la posición cuadrática promedio depende del tiempo en forma cuadrática, por lo tanto, el electrón se deslocaliza en el espacio real, como se puede notar este resultado concuerda con el resultado clásico antes obtenido.
 
 Para el caso tangente hiperbólica se encontró que existen restricciones para los parámetros de \textit{Hopping}, frecuencia forzante y constantes en la expansión de la tangente hiperbólica. Estas restricciones garantizan soluciones no triviales y de estabilidad para el sistema. Tomando en cuenta las restricciones se halló que el electrón presentará un movimiento periódico y acotado alrededor del origen, se hizo un estudio de la densidad de probabilidad tanto en el espacio de momentos como en espacio real, resultando que la densidad de probabilidad del electrón en el espacio de momentos será una onda plana con frecuencias bien definidas cuya intensidad tiene una dependencia temporal periódica de periodo $\tau=2\pi/\omega$. Se halló también que en el espacio real existen tres posiciones posibles para el electrón, una en el origen y dos simétricas con el origen, se halló que la probabilidad del electrón en estos puntos presenta una dependencia temporal periódica de periodo $\tau=2\pi/ \omega$. En este periodo el electrón pasa de estar totalmente localizado en el origen, a tener probabilidad no nula de encontrarse en los otros dos puntos probables antes dichos. Se encuentra entonces que el electrón presenta un comportamiento igual al predicho en el calculo semiclásico, donde también se encontró que existen tres puntos de equilibrio para el movimiento.
 
