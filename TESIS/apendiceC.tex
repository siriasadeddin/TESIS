\chapter{Obtención de la formula recursiva de Floquet-Magnus}\label{apendice:C}

Usando las definiciones expuestas en el capítulo \ref{cap:8}, se seguirá el procedimiento expuesto por Maricq en 1982 \cite{maricq} para obtener la formula recursiva de Floquet-Magnus.

Se substituye \ref{eq:8.3} en \ref{eq:8.1} y se obtiene:

\begin{equation}\label{eq:D.1}
    \frac{dP(t)}{dt}=-iH(t)P(t)+iP(t)\overline{H}
\end{equation}

Se hace además la sustitución $H(t)  \rightarrow \lambda H(t)$ y se introducen las ecuaciones \ref{eq:8.4} y \ref{eq:8.5} en \ref{eq:8.6}, separando las ecuaciones por orden de lambda.

\begin{equation}\label{eq:D.2}
    \sum^{\infty}_{n=0}\lambda^n\frac{dP_n(t)}{dt}=-i\lambda H(t)\sum^{\infty}_{n=0}\lambda^nP_n(t)+i\sum^{\infty}_{k=0}\lambda^kP_k(t)\sum^{\infty}_{m=0}\lambda^m\overline{H_m}
\end{equation}

A orden de $\lambda^n$

\begin{equation}\label{eq:D.3}
    \lambda^n\frac{dP_n(t)}{dt}=-i H(t)\sum^{\infty}_{n=0}\lambda^{n+1}P_n(t)+i\sum^{\infty}_{k=0}\sum^{\infty}_{m=0}\lambda^{k+m}P_k(t)\overline{H_m}
\end{equation}

Se hacen los cambios $n'=n+1$, $n=k+m$

\begin{equation}\label{eq:D.4}
    \lambda^n\frac{dP_n(t)}{dt}=-i\left[ H(t)\lambda^{n'}P_{n'-1}(t)-\sum^{n}_{k=0}\lambda^{n}P_k(t)\overline{H_{n-k}}\right]
\end{equation}

Haciendo $\lambda=1$ y $n'=n$ se obtiene sin pérdida de generalidad

\begin{equation}\label{eq:D.5}
    \frac{dP_n(t)}{dt}=-i\left[ H(t)P_{n-1}(t)-\sum^{n-1}_{k=1}P_k(t)\overline{H_{n-k}}-P_{0}(t)\overline{H_n}-P_n(t)H_0(t)\right]
\end{equation}

\begin{equation}\label{eq:D.6}
    \frac{dP_n(t)}{dt}=-i\left[ H(t)P_{n-1}(t)-\sum^{n-1}_{k=1}P_k(t)\overline{H_{n-k}}-\overline{H_n}\right]
\end{equation}

Integrando a ambos lados de la ecuación se obtiene: 

\begin{equation}\label{eq:D.7}
    P_n(t)=-i\int^{t}_{0}\left[ H(t')P_{n-1}(t')-\sum^{n-1}_{k=1}P_k(t')\overline{H}_{n-k}-\overline{H}_n\right]
\end{equation}

Se integra ahora la ecuación \ref{eq:D.7} de en un periodo $\tau$:

\begin{equation}\label{eq:D.8}
    \int^{\tau}_{0}\frac{dP_n(t)}{dt}=-i\int^{\tau}_{0}\left[ H(t)P_{n-1}(t')-\sum^{n-1}_{k=1}P_k(t')\overline{H_{n-k}}-\overline{H_n}\right]
\end{equation}

Resolviendo \ref{eq:D.8} se puede despejar $\overline{H_n}$:

\begin{equation}\label{eq:D.9}
    P_n(\tau)-P_n(0)=-i\int^{\tau}_{0}\left[ H(t)P_{n-1}(t')-\sum^{n-1}_{k=1}P_k(t')\overline{H_{n-k}}\right]-\overline{H_n}\tau
\end{equation}

Apelando a la periodicidad de $P_n(t)$ el lado izquierdo de la ecuación se anula por lo que se obtiene el siguiente resultado para $\overline{H_n}$

\begin{equation}\label{eq:D.10}
    \overline{H_n}=\frac{-i}{\tau}\int^{\tau}_{0}\left[ H(t)P_{n-1}(t')-\sum^{n-1}_{k=1}P_k(t')\overline{H_{n-k}}\right]
\end{equation}

\section{Convergencia de las series}

La integración consecutiva de $H(t)$, junto con su condición de estar acotado nos ayudará a determinar la convergencia de $U(t)$ y en consecuencia la convergencia de la expansión de Floquet-Magnus.

\begin{equation}\label{eq:D.11}
   \int^{t}_0 H(t)dt \leq \int^{t}_0 dt = t
\end{equation}

\begin{equation}\label{eq:D.12}
    \int^{t}_0H(t_k)\dots \int^{t}_0H(t_1)dt_1 \dots dt_k \leq \frac{t^k}{k!}
\end{equation}

Un procedimiento análogo (para la cota inferior -1)  deja el resultado:

\begin{equation}\label{eq:D.13}
    -\frac{t^k}{k!} \leq \int^{t}_0H(t_k)\dots \int^{t}_0H(t_1)dt_1 \dots dt_k
\end{equation}

Por lo tanto $U_k-U_{k-1}$ tendrá la siguiente forma:

\begin{equation}\label{eq:D.14}
    \norm{U_k-U_{k-1}}\leq \frac{\lambda^kt^k}{k!}
\end{equation}

Se encuentra entonces que a medida que $k$ crece el error disminuye drásticamente, y en consecuencia $U_k$ converge. Por ende, $e^{-i\overline{H}\tau}$ debe converger, entonces $\overline{H}$ converge. En consecuencia la solución de Floquet  $U(t)=P(t)e^{-i\overline{H}t}$ converge. 
